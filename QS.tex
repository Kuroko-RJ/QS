\documentclass[draft]{article}

\usepackage[UTF8]{ctex}
\usepackage[a4paper,scale=.7]{geometry}
\usepackage{amsmath,amsfonts,mathrsfs,amssymb,ntheorem,extarrows,graphicx,enumerate,braket,mathtools,xcolor,listings,syntonly}
%\usepackage{showlabels}


%\numberwithin{equation}{section}

\newcommand*\diff{\mathop{}\!\mathrm{d}}
\newcommand*\Diff[1]{\mathop{}\!\mathrm{d^#1}}
\newcommand*\bd{\mathop{}\!\text{\dj}}

\newcommand\der[1]{\frac{\diff}{\diff #1}}
\newcommand*\Der[2]{\frac{\diff #1}{\diff #2}}
\newcommand\nder[2]{\frac{{\diff}^#1}{{\diff #2}^#1}}
\newcommand*\nDer[3]{\frac{{\diff}^#1 #2}{{\diff #3}^#1}}
\newcommand\parder[1]{\frac{\partial}{\partial #1}}
\newcommand*\parDer[2]{\frac{\partial #1}{\partial #2}}
\newcommand\nparder[2]{\frac{\partial^#1}{{\partial #2}^#1}}
\newcommand*\nparDer[3]{\frac{\partial^#1#2}{{\partial #3}^#1}}%定义导数的快捷输入

\renewcommand*\geq\geqslant
\renewcommand*\leq\leqslant
\renewcommand*\ge\geqslant
\renewcommand*\le\leqslant

\renewcommand*\phi\varphi
\renewcommand*\epsilon\varepsilon
\renewcommand*\deg{^\circ}

\renewcommand*{\(}{\left(}
\renewcommand*{\)}{\right)}

\newcommand\equs[1]{\left\{\, #1\right.}

\newcommand*\ee{\mathrm{e}}
\newcommand*\ii{\mathrm{i}}
\newcommand*\vXi{\varXi}

\newcommand\vcbd[1]{\boldsymbol{#1}}

\newcommand*\unit[1]{\,\mathrm{#1}}
\newcommand*\e[1]{\times 10^{#1}}
\newcommand*\myemph[2]{\emph{#1}(\emph{#2})}

\newcommand*\taghere{\refstepcounter{equation}\tag{\theequation}}

\newcommand*\xde{Schrödinger}
\newcommand*\xdefc{\xde 方程}
\newcommand*\dbly{de Broglie}
\newcommand\expe[1]{\left\langle #1 \right\rangle}

\newcommand*\operaise{\hat{a}_+}
\newcommand*\opelower{\hat{a}_-}
\newcommand*\opeladder{\hat{a}_\pm}
\newcommand*\commutator[2]{\left[\hat{#1},\hat{#2}\right]}
\newcommand*\ope[1]{\hat{#1}}
\newcommand*\opeadj[1]{\hat{#1}^\dag}

\newcommand*\ex{\vcbd{e_x}}
\newcommand*\ey{\vcbd{e_y}}
\newcommand*\ez{\vcbd{e_z}}
\newcommand*\er{\vcbd{e_r}}
\newcommand*\etheta{\vcbd{e_\theta}}
\newcommand*\ephi{\vcbd{e_\phi}}

\newcommand*\hilbert{\mathcal{H}}

\newtheorem*{props}{命题}
{
	\theorembodyfont{\normalfont}
	\newtheorem{exam}{例}
}
{
\theoremstyle{plain}
\newtheorem{exer}{}
}
{
\theoremstyle{nonumberplain}
\theorembodyfont{\normalfont}
\newtheorem{ans}{解:}
}
{
	\theoremstyle{nonumberplain}
	\theorembodyfont{\normalfont}
	\newtheorem{proof}{证:}
}

\lstset{
	basicstyle          =   \sffamily,          % 基本代码风格
	keywordstyle        =   \bfseries,          % 关键字风格
	flexiblecolumns,                % 别问为什么,加上这个
	numbers             =   left,   % 行号的位置在左边
	showspaces          =   false,  % 是否显示空格,显示了有点乱,所以不现实了
	numberstyle         =   \ttfamily,    % 行号的样式,小五号,tt等宽字体
	showstringspaces    =   false,
	frame               =   lrtb,   % 显示边框
	breaklines			=	true,
	backgroundcolor=\color{black!5},
}%定义代码抄录环境

\DeclareMathOperator{\li}{Li}

\title{abc}
\author{龍翔九州}
\date{\today}

\begin{document}
	\maketitle
	\section{理想气体的统计分布}
	本文主要考虑玻色气体和费米气体的统计性质。在自由气体近似下构成气体的玻色子 及费米子的能级与经典理想气体并没有什么区别,它们的统计特性主要来自于多个玻色子 或费米子带来的全同粒子效应以及泡利不相容原理。我们首先考虑单个粒子的能级,将粒子近似为在体积$ V $的范围内运动,当$ V\gg \lambda^3 $的时候能级视为准连续,其中$ \lambda=2\pi\hbar/p $是粒子的德布罗意波长,那么粒子的能级$E=p^2/2m$. 粒子在能级$ E $附近的态密度$ g(E) $为\footnote{\eqref{first}式中分子的$4\pi p^2\diff p$代表$ \vcbd{p}- $空间中能量为$ E $至$ E+\diff E $的区域所占据的球壳体积,自然有$ E=p^2/2m $及$ \diff E=p\diff p/m $。分母的$ (2\pi\hbar)^3/V $代表每个$ \vcbd{p} $态所占据的体积,两个体积相除就得到了$ E $至$ E+\diff E $范围内的量子态数目。}
	\[\label{first}g(E)\diff E=\frac{4\pi p^2\diff p}{(2\pi\hbar)^2/V}\times g=g\frac{V}{\sqrt{2}\pi^2\hbar^3}m^{\frac 32}\sqrt E\diff E\taghere\]
	常数$ g $代表由于粒子自旋引入的简并度。式\eqref{first}即
	\[\label{DensityOfState}g(E)=g\frac{m^{\frac 32}V}{\sqrt{2}\pi^2\hbar^3}\sqrt E\taghere\]
	若能知道粒子在能级上的分布$ f(E) $, 即能级$ E $上量子态被粒子实际占据的比例,就能得到气体的热力学性质,例如
	\begin{subequations}\label{StatisticalQuantity}
		\begin{align}
			N&=\int_0^\infty f(E)g(E)\diff E\\
			U&=\int_0^\infty Ef(E)g(E)\diff E
		\end{align}
	\end{subequations}
	\begin{subequations}\label{Distribution}
		对于玻色气体,其分布为
		\[\label{BoseDistribution}f_{\text{Bose}}(E)=\frac 1{\ee^{\frac{E-\mu}{kT}}-1}\taghere\]
		对于费米气体,其分布为
		\[\label{FermiDistribution}f_{\text{Fermi}}(E)=\frac{1}{\ee^{\frac{E-\mu}{kT}}+1}\taghere\]
	\end{subequations}
	式\eqref{Distribution}的推导教材上已经详细写出,不再重复。
	\section{玻色气体}
	现在理想气体的统计理论框架已经搭好了,只要把式\eqref{DensityOfState}和\eqref{BoseDistribution}哐哐往\eqref{StatisticalQuantity}里代就完成了……本应如此,我们先实际代入一下。玻色子自旋假设为0,这样就没有自旋带来的简并:
	\[\label{5}N=\frac{m^{\frac 32}V}{\sqrt{2}\pi^2\hbar^3}\int_0^\infty\frac{\sqrt{E}\diff E}{\ee^{(E-\mu)/kT}-1}\taghere\]
	作换元$ E\to kTx $,引入参数$ z=\ee^{\mu/kT} $,那么
	\[\label{6}N=\frac{m^{\frac 32}V}{\sqrt{2}\pi^2\hbar^3}(kT)^{\frac 32}\int_0^\infty\frac{\sqrt{x}\diff x}{z^{-1}\ee^x-1}=\frac{m^{\frac 32}V}{\sqrt{2}\pi^2\hbar^3}(kT)^{\frac 32}\cdot\frac{\sqrt\pi}{2}h_{\frac 32}(z)\taghere\]
	其中
	\[\label{int}h_\nu(z):=\frac 1{\Gamma(\nu)}\int_0^{\infty}\frac{x^{\nu-1}}{z^{-1}\ee^x-1}\diff x\taghere\]
	$ h_\nu(z) $并不是初等函数,但是可以表达为级数
	\[\label{series}h_\nu(z)=\sum_{n=1}^\infty\frac{z^n}{n^\nu}=\li_\nu(z)\taghere\]
	通过级数表达式我们发现$ h_\nu(z) $就是多对数函数$ \li_s(z) $。不过我们不太需要这么花里胡哨的名字,就还是叫它$ h_\nu(z) $好了。我们把$ h_\nu(z) $画出来,看看它是什么样的:
	\begin{figure}[h]
		\includegraphics[width=\linewidth]{fig1}\label{fig1}
	\end{figure}

	我们发现$ h_\nu(z) $只有在$ z\le 1 $的时候有定义,在$ z>1 $的时候不仅级数\eqref{series}发散至$ \infty $,积分\eqref{int}本身也发散。并且按照级数\eqref{series}有
	\[h_\nu(1)=\sum_{n=1}^{\infty}\frac{1}{n^\nu}=\zeta(\nu)\taghere\]
	$ \zeta(\nu) $是另一个特殊函数,没别的了。对于玻色子来说我们只需要关心$ z>0 $的情况,毕竟$ \ee^{\mu/kT}>0 $.但是对于费米子我们会涉及到另一个积分
	\[g_\nu(z):=\frac{1}{\Gamma(\nu)}\int_0^\infty\frac{x^{\nu-1}}{z^{-1}\ee^x-1}\diff x=-h_\nu(-z)\taghere\]
	相当于把$ h_\nu(z) $在第四象限的部分旋转到第一象限,在上图里我们一并画了出来。
	
	把式\eqref{6}变形有
	\[\label{61}h_{\frac 32}(z)=\left(\frac{2\pi\hbar^2}{mkT}\right)^{\frac 32}\frac NV\taghere\]
	式\eqref{61}相当于一个方程,对一个给定温度$ T $的理想玻色子系统应该有一个确定的$ z=\ee^{\mu/kT} $也就是确定的化学势$\mu$. 看上去很完美,但是当$ T $比较小的时候式\eqref{61}的右边就超出$ h_{\frac 32}(z) $在0到 1上的最大值$ \zeta(3/2) $了。这时候玻色分布肯定还是对的,只不过式\eqref{6}左边的$ N $发生了变化。我们发现这时候总有
	\[\label{12}N\ge V\left(\frac{mkT}{2\pi\hbar^2}\right)^{\frac 32}h_{\frac 32}(1)\taghere\]
	问题出在式\eqref{5}右边的积分只考虑了处在$ E>0 $状态的玻色子。我们认为\eqref{12}右侧就是处于能级$ E>0 $的玻色子个数$ N_{>0} $,而左侧是总个数$ N $,两者的差就是处在$ E=0 $的玻色子个数$ N_{=0} $. 也就是说,在足够低的温度下会有非常多的玻色子被“冻结”在基态能级,只有一部分玻色子能够激发到$ E>0 $的能级。这就是玻色-爱因斯坦凝聚现象。直观的来讲这是由于低温$ T $下系统内含的热运动能量没有办法让所有玻色子都产生激发。
	
	我们定义使\eqref{12}式取等号的温度为临界温度$ T_C $
	\[T_C:=\frac{2\pi\hbar^2}{mk}\left(\frac 1{\zeta(3/2)}\frac NV\right)^{\frac 23}\taghere\]
	在$ T>T_C $的时候系统所有玻色子均被激发。而$ T<T_C $的时候有$ N_{>0} $个玻色子被激发
	\begin{subequations}\label{14}
	\[\label{14a}N_{>0}=V\left(\frac{mkT}{2\pi\hbar^2}\right)^{\frac 32}h_{\frac 32}(1)=N\left(\frac{T}{T_C}\right)^{\frac 32}\taghere\]
	剩下$ N_{=0} $个玻色子被冻结在基态
	\[N_{=0}=N\left(1-\left(\frac{T}{T_C}\right)^{\frac 32}\right)\taghere\]
	\end{subequations}
	这里$ N_{=0} $和$ N_{>0} $都是温度$ T $的函数,而$ N=N_{=0}+N_{>0} $是一个常量。系统的温度越低,能量就越低,被冻结在基态的玻色子越多,被激发的玻色子越少。
	
	接下来考虑$ T<T_C $时玻色子气体的内能,根据式\eqref{StatisticalQuantity}有
	\[\label{15}U=\frac{m^{\frac 32}V}{\sqrt{2}\pi^2\hbar^3}(kT)^{\frac 52}\times \frac{3\sqrt{\pi}}{4}h_{\frac 52}(z)=\frac 32VkT\left(\frac{mkT}{2\pi\hbar^2}\right)^{\frac 32}h_{\frac 52}(1)\taghere\]
	其中$ z=\ee^{\mu/kT} $在$ T<T_C $时取1. 将式\eqref{15}和\eqref{14a}相对比可以得到
	\[\frac{U}{NkT}=\frac 32\left(\frac{T}{T_C}\right)^{\frac 32}\frac{h_{5/2}(1)}{h_{3/2}(1)}\]
	也就是说
	\[U=\frac{3\zeta(5/2)}{2\zeta(3/2)}NkT\left(\frac{T}{T_C}\right)^{\frac 32}\approx0.770NkT\left(\frac{T}{T_C}\right)^{\frac 32}\taghere\]
	此时玻色气体的热容
	\[C_V=\parDer UT=\frac{5U}{2T}\taghere\]
	$ T_C $表征了玻色气体发生玻色-爱因斯坦凝聚的临界温度,在玻色子系统里$ T_C $往往远低于常温。例如Anderseon等人曾经对$ ^{87}\text{Rb} $原子降温,直到170nK才首次观察到$ ^{87}\text{Rb} $原子系统的玻色-爱因斯坦凝聚现象\footnote{Anderson, M. H., Ensher, J. R., Matthews, M. R., Wieman, C. E., and Cornell, E. A. (1995). Observation of Bose–Einstein condensation in a dilute atomic vapor. Science, 269, 198.}。在常温下$ T\gg T_C $, 在高温近似下玻色气体逐渐体现出经典统计特性。具体来说,在$ T>T_C $时仍有
	\[\label{18}h_{\frac 32}(z)=\frac NV\left(\frac{2\pi\hbar^2}{mkT}\right)^{\frac 32}=\zeta\left(\frac{T_C}{T}\right)^{\frac 32}\taghere\]
	式\eqref{18}右侧在高温下接近0,依级数表达式\eqref{series}有近似$ h_{\nu}(z)\sim z $,因此
	\[\mu=kT\ln\left[\frac NV\left(\frac{2\pi\hbar^2}{mkT}\right)^{\frac 32}\right]\sim-\frac 32kT\ln T\taghere\]
	因而玻色气体内能
	\[U=\frac 32VkT\left(\frac{mkT}{2\pi\hbar^2}\right)^{\frac 32}h_{\frac 52}(z)=\frac 32NkT\cdot \frac{h_{5/2}(z)}{h_{3/2}(z)}\taghere\]
	由级数\eqref{series}有
	\[\frac{h_{5/2}(z)}{h_{3/2}(z)}\approx\frac{z+\dfrac{z^2}{2^{5/2}}}{z+\dfrac{z^2}{2^{3/2}}}\approx1-\frac{z}{4\sqrt{2}}\]
	因此
	\[U\approx\frac 32NkT\left(1-\frac{\zeta(3/2)}{4\sqrt{2}}\left(\frac{T_C}{T}\right)^{\frac 32}\right)\approx\frac 23NkT\taghere\]
	热容
	\[C_V\approx\frac 32Nk\left(1+\frac{\zeta(3/2)}{8\sqrt{2}}\left(\frac{T_C}{T}\right)^{\frac 32}\right)\approx\frac 32Nk\taghere\]
	这说明常温下玻色气体退化为理想气体,量子性被经典性掩盖。
	
	另一方面,引入热德布罗意波长$ \lambda_{\text{th}} $:
	\[\lambda_{\text{th}}=\sqrt{\frac{2\pi\hbar^2}{mkT}}\taghere\]
	将\eqref{12}改写为
	\[\label{12'}\frac NV\ge\frac{\zeta(3/2)}{\lambda_{\text{th}}^2}\tag{\ref{12}$ ' $}\]
	\eqref{12'}式也说明当玻色气体致密到平均粒子密度$ N/V $大于某个值时,粒子间平均间距小于粒子的波长,不同粒子的波包相互重叠形成一个大的波包,这些粒子就会凝聚在同一个状态\footnote{在知乎上有讲得更好的定性分析,我写的东西仅供参考(}。
	%参考https://www.zhihu.com/question/440094636/answer/1685768832
	%https://www.zhihu.com/question/339890379/answer/880748843
	\section{费米气体}
	费米气体算起来简单点,至少不用像玻色气体那样分段。首先把\eqref{StatisticalQuantity}式写出来,以自旋1/2的费米子为例,自旋带来的简并度为2:
	\begin{subequations}\label{24}\begin{align}
		N&=\frac{\sqrt{2}V}{\pi^2\hbar^3}m^{\frac{3}{2}}\int_0^\infty\frac{\sqrt{E}\diff E}{\ee^{\frac{E-\mu}{kT}}+1}=\frac{V}{\sqrt{2}}\left(\frac{mkT}{\pi\hbar^2}\right)^{\frac{3}{2}}g_{\frac 32}(z)\label{24a}\\
		U&=\frac{\sqrt{2}V}{\pi^2\hbar^3}m^{\frac{3}{2}}\int_0^\infty\frac{E^{\frac 32}\diff E}{\ee^{\frac{E-\mu}{kT}}+1}=\frac{3\sqrt{2}VkT}{4}\left(\frac{mkT}{\pi\hbar^2}\right)^{\frac 32}g_{\frac 52}(z)
	\end{align}\end{subequations}
	其中
	\[g_\nu(z):=\int_0^\infty\frac{x^{\nu-1}}{z^{-1}\ee^x+1}\diff x=-h_\nu(-z)\]
	之前图中黄色曲线即是$ g_\nu(z) $的图象。下图画出了不同$ \nu $值下$ g_\nu(z) $的图象。$ \nu $越大$ h_\nu(z) $也越大,曲线远离$ x $轴。
	\begin{figure}[h]
		\includegraphics[width=\linewidth]{fig2}
	\end{figure}
	
	先来考虑低温$ T\to 0 $的情况,由\eqref{24a}变形得
	\[\label{25}g_{\frac 32}(z)=\frac{\sqrt 2N}{V}\left(\frac{\pi\hbar^2}{mkT}\right)^{\frac 32}\to+\infty\taghere\]
	所幸$ z\to\infty $的时候$ g_\nu(z)\to\infty $,这保证了\eqref{25}总是有解的,且$ z=\ee^{\mu/kT}\to\infty $. 这里不加证明地给出$ g_\nu(z) $在$ z\to\infty $时的渐进表达式:
	\[\label{26}g_\nu(z)\approx\frac{(\ln z)^\nu}{\Gamma(\nu+1)}+\frac{\pi^2}{6}\frac{(\ln z)^{\nu-2}}{\Gamma(\nu-1)}\quad(z\to\infty)\taghere\]
	\eqref{25}式对每个温度$ T $都唯一确定了化学势$ \mu(T) $. 我们想从\eqref{25}解出$ z $从而解出$ \mu $,为此先把\eqref{26}右侧第一项代入\eqref{25}式中得到
	\[\frac{4(\ln z)^{\frac 32}}{3\sqrt{\pi}}=\frac{\sqrt 2N}{V}\left(\frac{\pi\hbar^2}{mkT}\right)^{\frac 32}\to+\infty\taghere\]
	也就是说
	\[\label{28}\mu(T)=kT\ln z\approx\frac{\hbar^2}{2m}\left(3\pi^2\frac NV\right)^{\frac 23}\taghere\]
	\eqref{28}式是$ T\to 0 $时对$ \mu $的最低阶近似,表明$ T\to 0 $时费米气体的化学势趋于一定值,定义\emph{费米能级}
	\[\mu(0):=\frac{\hbar^2}{2m}\left(3\pi^2\frac NV\right)^{\frac 23}\taghere\]
	为$ T=0 $时的化学势。费米能级的概念稍后还会涉及。利用费米能级$ \mu(0) $将\eqref{25}改写为
	\[\label{25'}g_{\frac 32}(z)=\frac{4}{3\sqrt{\pi}}\left(
	\frac{\mu(0)}{kT}\right)^{\frac 32}\tag{\ref{25}$ ' $}\]
	我们继续计算$ \mu $的下一级近似,将\eqref{26}式右边两项均代入\eqref{25'}式得到
	\[\left(\frac{\mu(0)}{kT}\right)^{\frac 32}=\left(\frac{\mu}{kT}\right)^{\frac 32}\left(1+\frac{\pi^2}8\left(\frac{kT}{\mu}\right)^2\right)\]
	也就是说
	\[\label{30}\frac\mu{\mu(0)}=\left(1+\frac{\pi^2}8\left(\frac{kT}{\mu}\right)^2\right)^{-\frac 23}\approx1-\frac{\pi^2}{12}\left(\frac{kT}{\mu(0)}\right)^2\taghere\]
	\eqref{30}式表明低温下化学势$ \mu $随温度上升会下降。
	以及
	\begin{align*}
		g_\nu(z)&\approx\frac 1{\Gamma(\nu+1)}\left(\frac{\mu}{kT}\right)^\nu\left(1+\frac{\pi^2}{6}\nu(\nu-1)\left(\frac{kT}{\mu}\right)^2\right)\\
		&=\frac 1{\Gamma(\nu+1)}\left(\frac{\mu(0)}{kT}\right)^\nu\left(1-\frac{\pi^2}{12}\left(\frac{kT}{\mu(0)}\right)\right)^\nu\left(1+\frac{\pi^2}{6}\nu(\nu-1)\left(\frac{kT}{\mu(0)}\right)^2\right)\\
		&\approx\frac 1{\Gamma(\nu+1)}\left(\frac{\mu(0)}{kT}\right)^\nu\left(1+\frac{\pi^2}{12}(2\nu^2-3\nu)\left(\frac{kT}{\mu(0)}\right)^2\right)\taghere
	\end{align*}
	也就是说
	\begin{align*}
		g_{\frac 32}(z)&=\frac{1}{\Gamma(5/2)}\left(
		\frac{\mu(0)}{kT}\right)^{\frac 32}\\
		g_{\frac 52}(z)&=\frac{1}{\Gamma(7/2)}\left(
		\frac{\mu(0)}{kT}\right)^{\frac 52}\left(1+\frac{5\pi^2}{12}\left(\frac{kT}{\mu(0)}\right)^2\right)
	\end{align*}
	进一步可以算出内能$ U $, 由\eqref{24}式有
	\begin{align*}
	\frac{U}{NkT}&=\frac 32\frac{g_{5/2}(z)}{g_{3/2}(z)}\approx\frac 32\times\frac 25\frac{\mu(0)}{kT}\left(1+\frac{5\pi^2}{12}\left(\frac{kT}{\mu(0)}\right)^2\right)
	\end{align*}
	也就是说
	\[\label{FeimiGasEnergy}U=\frac 35N\mu(0)+\frac{\pi^2}{4}\frac{N}{\mu(0)}(kT)^2\taghere\]
	内能\eqref{FeimiGasEnergy}右侧第一项为常数,是由于泡利不相容原理带来的内禀能量。第二项正比于$ T^2 $,因此可以得到低温下费米气体热容
	\[C_V=\parDer UT=\frac{\pi^2}{2}Nk\frac{kT}{\mu(0)}\propto T\taghere\]
\end{document}