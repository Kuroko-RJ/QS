\usepackage[UTF8]{ctex}
\usepackage[a4paper,scale=.7]{geometry}
\usepackage{amsmath,amsfonts,mathrsfs,amssymb,ntheorem,extarrows,graphicx,enumerate,braket,mathtools,xcolor,listings,syntonly,multicol}
%\usepackage{showlabels}


%\numberwithin{equation}{section}

\newcommand*\diff{\mathop{}\!\mathrm{d}}
\newcommand*\Diff[1]{\mathop{}\!\mathrm{d^#1}}
\newcommand*\bd{\mathop{}\!\text{\dj}}

\newcommand\der[1]{\frac{\diff}{\diff #1}}
\newcommand*\Der[2]{\frac{\diff #1}{\diff #2}}
\newcommand\nder[2]{\frac{{\diff}^#1}{{\diff #2}^#1}}
\newcommand*\nDer[3]{\frac{{\diff}^#1 #2}{{\diff #3}^#1}}
\newcommand\parder[1]{\frac{\partial}{\partial #1}}
\newcommand*\parDer[2]{\frac{\partial #1}{\partial #2}}
\newcommand\nparder[2]{\frac{\partial^#1}{{\partial #2}^#1}}
\newcommand*\nparDer[3]{\frac{\partial^#1#2}{{\partial #3}^#1}}%定义导数的快捷输入

\renewcommand*\geq\geqslant
\renewcommand*\leq\leqslant
\renewcommand*\ge\geqslant
\renewcommand*\le\leqslant

\renewcommand*\phi\varphi
\renewcommand*\epsilon\varepsilon
\renewcommand*\deg{^\circ}

\renewcommand*{\(}{\left(}
\renewcommand*{\)}{\right)}

\newcommand\equs[1]{\left\{\, #1\right.}

\newcommand*\ee{\mathrm{e}}
\newcommand*\ii{\mathrm{i}}
\newcommand*\vXi{\varXi}

\newcommand\vcbd[1]{\boldsymbol{#1}}

\newcommand*\unit[1]{\,\mathrm{#1}}
\newcommand*\e[1]{\times 10^{#1}}
\newcommand*\myemph[2]{\emph{#1}(\emph{#2})}

\newcommand*\taghere{\refstepcounter{equation}\tag{\theequation}}

\newcommand*\xde{Schrödinger}
\newcommand*\xdefc{\xde 方程}
\newcommand*\dbly{de Broglie}
\newcommand\expe[1]{\left\langle #1 \right\rangle}

\newcommand*\operaise{\hat{a}_+}
\newcommand*\opelower{\hat{a}_-}
\newcommand*\opeladder{\hat{a}_\pm}
\newcommand*\commutator[2]{\left[\hat{#1},\hat{#2}\right]}
\newcommand*\ope[1]{\hat{#1}}
\newcommand*\opeadj[1]{\hat{#1}^\dag}

\newcommand*\ex{\vcbd{e_x}}
\newcommand*\ey{\vcbd{e_y}}
\newcommand*\ez{\vcbd{e_z}}
\newcommand*\er{\vcbd{e_r}}
\newcommand*\etheta{\vcbd{e_\theta}}
\newcommand*\ephi{\vcbd{e_\phi}}

\newcommand*\hilbert{\mathcal{H}}

\newtheorem*{props}{命题}
{
	\theorembodyfont{\normalfont}
	\newtheorem{exam}{例}
}
{
\theoremstyle{plain}
\newtheorem{exer}{}
}
{
\theoremstyle{nonumberplain}
\theorembodyfont{\normalfont}
\newtheorem{ans}{解:}
}
{
	\theoremstyle{nonumberplain}
	\theorembodyfont{\normalfont}
	\newtheorem{proof}{证:}
}

\newenvironment{Figure}
{\par\medskip\noindent\minipage{\linewidth}}
{\endminipage\par\medskip}

\lstset{
	basicstyle          =   \sffamily,          % 基本代码风格
	keywordstyle        =   \bfseries,          % 关键字风格
	flexiblecolumns,                % 别问为什么,加上这个
	numbers             =   left,   % 行号的位置在左边
	showspaces          =   false,  % 是否显示空格,显示了有点乱,所以不现实了
	numberstyle         =   \ttfamily,    % 行号的样式,小五号,tt等宽字体
	showstringspaces    =   false,
	frame               =   lrtb,   % 显示边框
	breaklines			=	true,
	backgroundcolor=\color{black!5},
}%定义代码抄录环境